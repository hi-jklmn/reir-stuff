 % Skeleton report for Assignment S3, in Reiknirit, fall 2017, at Reykjavik University
%  (c) 2017, Magnus M. Halldorsson
%
% Based on an assignment at Princeton University, (c) Kevin Wayne
%
% Format is based on the following document:
% A basic LaTeX document for a handin with a standard RU title page
%  (c) 2013, Tómas Ken Magnússon, 

% If you want the title to appear on a separate page, change notitlepage to titlepage
\documentclass[11pt,a4paper,notitlepage]{article}
\usepackage[utf8]{inputenc}
\usepackage[T1]{fontenc}
% If your hand-in is in icelandic change english to icelandic
% Note: This has nothing to do with Icelandic characters, they
% can always be used. This just tells other packages what
% language you are using and changes the hyphenation used by LaTeX
% If icelandic is selected2 a shorthand, "` and "', is also included
% for Icelandic quotation marks. They can also obtained by using
% ,, and ``
\usepackage[english]{babel}
\usepackage{amsmath, amsthm, amssymb, amsfonts}
\usepackage{graphicx}
\usepackage{enumerate}
% To use the whole A4-page
% See: ftp://ftp.tex.ac.uk/tex-archive/macros/latex/contrib/geometry/geometry.pdf
% and http://en.wikibooks.org/wiki/LaTeX/Document_Structure
\usepackage{geometry}
% For header and footer
% See: ftp://ctan.tug.org/tex-archive/macros/latex/contrib/ fancyhdr/fancyhdr.pdf
% and http://en.wikibooks.org/wiki/LaTeX/Document_Structure
\usepackage{fancyhdr}
% For prettier tables
% See: http://ctan.mackichan.com/macros/latex/contrib/booktabs/booktabs.pdf
% and  http://en.wikibooks.org/wiki/LaTeX/Tables
%\usepackage{booktabs}
\usepackage{listings} % For code listing
\usepackage{url}
% \usepackage{sagetex}
\usepackage{hyperref}
\usepackage{caption}
\usepackage[usenames,dvipsnames,svgnames,table]{xcolor}
\usepackage{subfig}  % To fit two tables side-by-side

%%%%%%%%%%%%%%%%%%%%%%%%%%%%%%%%%%%%%%%%%%%%%%%%%%%%%%%%%%%%%
%                        Setup
%%%%%%%%%%%%%%%%%%%%%%%%%%%%%%%%%%%%%%%%%%%%%%%%%%%%%%%%%%%%%

% Set the margins of the paper. By default LaTeX uses huge margins
\geometry{includeheadfoot, margin=2.5cm}
% you can also use
% \geometry{a4paper}
% End of margins setup

% Settings for listings
\lstset{language=Java,numbers=left,backgroundcolor=\color{light-gray},
        basicstyle=\scriptsize\ttfamily,frame=single,tabsize=4,
        captionpos=t, numbers=left,
        keywordstyle=\color{javapurple}\bfseries,
        stringstyle=\color{javared},
        commentstyle=\color{javagreen},
        morecomment=[s][\color{javadocblue}]{/**}{*/},}
% End of settings for listings

% Custom colors for listings
\definecolor{light-gray}{gray}{0.95}
\definecolor{javared}{rgb}{0.6,0,0} % for strings
\definecolor{javagreen}{rgb}{0.25,0.5,0.35} % comments
\definecolor{javapurple}{rgb}{0.5,0,0.35} % keywords
\definecolor{javadocblue}{rgb}{0.25,0.35,0.75} % javadoc
% End of custom colors for listings


% Fill in any relevant information
% Leave the fields inside the {} empty if they do not apply
\newcommand{\semester}{Fall 2017}
\newcommand{\coursename}{Reiknirit}
\newcommand{\courseid}{T-301-REIR}
\newcommand{\assignment}{S4: WordNet}
\newcommand{\problemtitle}{Problem}
\newcommand{\dateofcompilation}{\today}

%% Information about you --- FILL THIS OUT
\newcommand{\ssn}{kt. 123456-7890}              %%%  CHANGE THIS """
\newcommand{\group}{1}
\newcommand{\teachingassistant}{TA: Eiríkur Fjalar}   %%%  CHANGE THIS """
\newcommand{\students}{
    Halli og Laddi                             %%%  CHANGE THIS """
}
\newcommand{\studentemail}{
    myEmails@ru.is                            %%%  CHANGE THIS """
}

% Setup header and footer
% Headers
\pagestyle{fancy} % To get the header and footer
\chead{\small \textsc{\assignment}}
\rhead{\small \textsc{\coursename}}
\lhead{\small \textsc{\studentemail}}
% Footers
%\lfoot{Left footer text}
%\cfoot{\thepage} % This is the default behaviour
%\rfoot{Right footer text}


% Custom dot for itemize
\renewcommand{\labelitemi}{$\cdot$}

% If you don't want a line below the header or above the footer,
% change the appropriate header/footerrulewidth to 0pt
\setlength{\headheight}{15.2pt} % This is set to avoid a warning
\renewcommand{\headrulewidth}{0.4pt}
\renewcommand{\footrulewidth}{0.4pt}
% End of header and footer setup

% Setup Problem/Solution environments // You probably don't need this
%\theoremstyle{plain}
%\newtheorem{problem}{Dæmi}
%\theoremstyle{remark}
%\newtheorem*{solution}{Lausn}
% End of Problem/Solution environments setup
%\theoremstyle{plain}
%\newtheorem*{proposition}{Proposition}

\DeclareCaptionLabelFormat{andtable}{#1~#2  \&  \tablename~\thetable}

% Custom problem (so you can provide the problem name)
%\newenvironment{cproblem}[1]{\begin{trivlist}
%\item[\hskip \labelsep {\bfseries \problemtitle}\hskip \labelsep {\bfseries#1.}]\begin{itshape}}{\end{itshape}\end{trivlist}}
% End of Problem/Solution environments setup

% The title page
\newcommand{\maketitlepage}[1]
{
    \begin{titlepage}

        \begin{center}
            \includegraphics[width=0.55\textwidth]{./rulogo.png}\\[1.5cm]

            \textsc{\huge \semester}\\[0.8cm]

            {\textsc{\Huge \courseid, \coursename}}\\[0.4cm]
            \textsc{\LARGE }\\[2.5cm]

            \textbf{\textsc{\Huge #1}}\\[3cm]


            \textsc{\huge \students}\\[0.4cm]
            \textsc{\LARGE \ssn}\\[0.4cm]
            \textsc{\LARGE Group \group}\\[1cm]
            \textsc{\Large \dateofcompilation}


        \end{center}

        \vfill

        % You may also want to add the name of your teaching assistant
        \begin{flushleft}
            \textsc{\Large \teachingassistant}
        \end{flushleft}

    \end{titlepage}
}
\newcommand{\command}[1]{\texttt{\textbackslash{}#1}}

\newcommand{\explanation}[1]{}  %% Use this when turning in the report
%\newcommand{\explanation}[1]{\begin{quote}\emph{#1} \end{quote}}  %% Use this to include directions

%%%%%%%%%%%%%%%%%%%%%%%% END OF SETUP %%%%%%%%%%%%%%%%%%%%%%%%


\begin{document}
% Create the title page
    \maketitlepage{\assignment}

\explanation{Directions on performing the assignment are showed here in italics (like this). These should not be included in the report you submit.}


\section{Data Structures}

\subsection*{synsets}

\explanation{Describe concisely the data structure(s) you used to store the 
information in synsets.txt. Why did you make this choice?}

\subsection*{hypernyms}

\explanation{Describe concisely the data structure(s) you used to store the 
information in hypernyms.txt. Why did you make this choice?}

\section{Algorithms}

\subsection*{Rooted check}

\explanation{
  Describe concisely the algorithm you used to check if the digraph 
  is rooted and the algorithm you used to check if the digrah is a DAG.  
  What is the order of growth of the best case 
  running time as a function of the number of vertices V and the 
  number of edges E in the digraph? And what is the order of growth 
  of the worst case running time?}

\explanation{
 Be careful! It is very easy to get these wrong. Keep in mind
  what the 'best case' and 'worst case' entail. Don't forget about
  the fact that starting a breadth first search in Java means 
  initializing edgeTo[] arrays, etc.
}

\begin{table}[htbp]
\renewcommand{\arraystretch}{2}
%  \small
%\baselineskip 1.5\baselineskip
  \centering
  \caption{!Insert caption!}
        \label{tab:table1}
        \begin{tabular}{l| c | c }
         & \qquad \emph{best case} \qquad & \qquad \emph{worst case} \qquad \\
        \hline
        \texttt{rooted check} & & \\\hline
        \texttt{DAG check} & & \\
        \hline
        \end{tabular}
\end{table}


\subsection{SAP}
\explanation{
  Describe concisely your algorithm to compute the shortest ancestral
  path in SAP.java? What is the order of growth of the worst-case
  running time of your methods as a function of the number of
  vertices $V$ and the number of edges $E$ in the digraph? What is the
  order of growth of the best-case running time?}



\begin{table}[htbp]
\renewcommand{\arraystretch}{2}
%  \small
%\baselineskip 1.5\baselineskip
  \centering
  \caption{!Insert caption!}
        \label{tab:table1}
        \begin{tabular}{l| c | c }
         \emph{method} & \qquad \emph{best case} \qquad & \qquad \emph{worst case} \qquad \\
        \hline
\texttt{length(int, int)}  & & \\\hline

\texttt{ancestor(int, int)}  & & \\\hline

\texttt{length(Iterable, Iterable)}  & & \\\hline

\texttt{ancestor(Iterable, Iterable)}  & & \\\hline

        \hline
        \end{tabular}
\end{table}


\subsection{Extra credit optimization}

\explanation{If you implemented an extra credit optimization describe it here.}


\section{About This Solution}

 
\subsection{Known Bugs / Limitations.}
% /******************************************************************************
\explanation{Known bugs / limitations. For example, if your program prints
  out different representations of the same line segment when there
 are 5 or more points on a line segment, indicate that here.}
%  *  Known bugs / limitations.
%  *****************************************************************************/

\subsection{Help Received}
% /******************************************************************************
\explanation{
Describe whatever help (if any) that you received.
Don't include readings, lectures, and classes, but do
include any help from people (including course staff, lab TAs,
classmates, and friends) and attribute them by name.}
%  *****************************************************************************/


\subsection{Problem Encountered}
% /******************************************************************************
\explanation{
Describe any serious problems you encountered.                    }
%  *****************************************************************************/



\subsection{Comments}
% /******************************************************************************
\explanation{
List any other comments here. Feel free to provide any feedback   
on how much you learned from doing the assignment, and whether    
you enjoyed doing it.}
% *****************************************************************************/




\end{document}

%%%% Optional material

% Example plot
%    \begin{figure}[!ht]
%        \label{fig:plot1}
%        \centering
%        \includegraphics[width=0.6\textwidth]{chart.pdf}
%        \caption{This is an imported figure}
%    \end{figure}

% Example reference:    See Listing \hyperref[lst:random]{1}.

    \pagebreak
    \section*{Appendix I: Source code listings}
    Optional.

    \subsection*{Acknowledgement}

    % Environment for listing code
    \begin{lstlisting}[caption={This is a caption.},label={lst:array1}]
public class This is a class {

    // This is a comment
    public static double thisIsAFunction(int N) {
        QuickUnionUF uf = new QuickUnionUF(N);
        return something;
    }
    public static void main(String[] args) { 
        int T = 50;
    }
} 
    \end{lstlisting}


    \pagebreak
    \section*{Appendix II}
    Optional.

    \listoftables
    \listoffigures
    \lstlistoflistings
    \bibliographystyle{plain}
    \bibliography{s1.bib}

\end{document}

